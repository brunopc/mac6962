\documentclass[reqno, a4paper, 11pt]{amsart}
\pdfoutput=1

\usepackage[brazil]{babel}
\usepackage[T1]{fontenc}
\usepackage[utf8]{inputenc}
\usepackage{textcomp}

\usepackage{ae}
\usepackage{aecompl}
\usepackage{lmodern}

\makeatletter
\let\origsection=\section \def\section{\@ifstar{\origsection*}{\mysection}}
\def\mysection{\@startsection{section}{1}\z@{.7\linespacing\@plus\linespacing}{.5\linespacing}{\normalfont\scshape\centering\S}}
\makeatother

\allowdisplaybreaks

\usepackage{fullpage}
\usepackage{setspace}
\usepackage{fancyhdr}
\pagestyle{fancy}

\lhead{}
\chead{}
\rhead{}
%\lfoot{}
\cfoot{}
%\rfoot{\today, \currenttime}
\fancyfoot[RO,LE]{\thepage}
\fancyfoot[LO,RE]{{\footnotesize\today, \currenttime}}
\renewcommand{\headrulewidth}{0pt}
\renewcommand{\footrulewidth}{0pt}

\usepackage{amsmath,amssymb,amsthm}
\usepackage{mathabx}\changenotsign
\usepackage{mathrsfs}
\usepackage{bbm}
\usepackage{dsfont}
\usepackage[babel]{microtype}
\usepackage{xcolor}
\usepackage[backref]{hyperref}
\hypersetup{
  colorlinks,
  linkcolor={red!60!black},
  citecolor={green!60!black},
  urlcolor={blue!60!black},
}
\usepackage{algpseudocode}

\usepackage{bookmark}

%%%% Let's use amsrefs http://www.ams.org/publications/authors/tex/amsrefs
\usepackage[abbrev,msc-links,backrefs]{amsrefs} 
\usepackage{doi}
\renewcommand{\doitext}{DOI\,}
\renewcommand{\PrintDOI}[1]{\doi{#1}}
\renewcommand{\eprint}[1]{\href{http://arxiv.org/abs/#1}{arXiv:#1}}
%%%%

\usepackage{tikz}
\usetikzlibrary{decorations.pathreplacing,calc}
\pgfdeclarelayer{background}
\pgfdeclarelayer{foreground}
\pgfsetlayers{background,main,foreground}

\newcommand\inj\hookrightarrow % Injective functions
\DeclareMathOperator\Var{Var} % Variance
\DeclareMathOperator\Cov{Cov} % Covariance

\usepackage[abbrev,msc-links,backrefs]{amsrefs}
\usepackage{doi}
\renewcommand{\doitext}{DOI\,}
\renewcommand{\PrintDOI}[1]{\doi{#1}}
\renewcommand{\eprint}[1]{\href{http://arxiv.org/abs/#1}{arXiv:#1}}

%\usepackage[english]{babel}
%\numberwithin{equation}{section}

\makeatletter
\def\@setdate{\datename\ \@date}
%\def\@setthanks{\def\thanks##1{\par##1}\thankses}
\def\@setaddresses{\par
  \nobreak \begingroup
\footnotesize
  \def\author##1{\nobreak\addvspace\bigskipamount}%
  \def\\{\unskip, \ignorespaces}%
  \interlinepenalty\@M
  \def\address##1##2{\begingroup
    \par\addvspace\bigskipamount\indent
    \@ifnotempty{##1}{(\ignorespaces##1\unskip) }%
    {\scshape\ignorespaces##2}\par\endgroup}%
  \def\curraddr##1##2{\begingroup
    \@ifnotempty{##2}{\nobreak\indent{\itshape Current address}%
      \@ifnotempty{##1}{, \ignorespaces##1\unskip}\/:\space
      ##2\par}\endgroup}%
  \def\email##1##2{\begingroup
    \@ifnotempty{##2}{\nobreak\indent{\itshape Endereço eletrônico}%
      \@ifnotempty{##1}{, \ignorespaces##1\unskip}\/:\space
      \ttfamily##2\par}\endgroup}%
  \def\urladdr##1##2{\begingroup
    \@ifnotempty{##2}{\nobreak\indent{\itshape URL}%
      \@ifnotempty{##1}{, \ignorespaces##1\unskip}\/:\space
      \ttfamily##2\par}\endgroup}%
  \addresses
  \endgroup
}
\def\datename{\textit{Data}:}
\makeatother

\usepackage{enumitem}
\def\rmlabel{\upshape({\itshape \roman*\,})}
\def\RMlabel{\upshape(\Roman*)}
\def\alabel{\upshape({\itshape \alph*\,})}
\def\Alabel{\upshape({\itshape \Alph*\,})}
\def\nlabel{\upshape({\itshape \arabic*\,})}
\def\AAlabel{\upshape({A\arabic*})}
\def\nplain{\upshape{\arabic*.}}
\def\aplabel{\upshape({\itshape \alph*\,$'$})}

\let\polishlcross=\l
\def\l{\ifmmode\ell\else\polishlcross\fi}

\def\tand{\ \text{and}\ }
\def\qand{\quad\text{and}\quad}
\def\qqand{\qquad\text{and}\qquad}

\let\emptyset=\varnothing
\let\setminus=\smallsetminus
\let\backslash=\smallsetminus
\let\subset\subseteq
\let\lg = \log
\let\log=\ln

\makeatletter
\def\moverlay{\mathpalette\mov@rlay}
\def\mov@rlay#1#2{\leavevmode\vtop{   \baselineskip\z@skip \lineskiplimit-\maxdimen
   \ialign{\hfil$\m@th#1##$\hfil\cr#2\crcr}}}
\newcommand{\charfusion}[3][\mathord]{
    #1{\ifx#1\mathop\vphantom{#2}\fi
        \mathpalette\mov@rlay{#2\cr#3}
      }
    \ifx#1\mathop\expandafter\displaylimits\fi}
\makeatother

\DeclareMathOperator{\dom}{{\rm dom}}

\newcommand{\dcup}{\charfusion[\mathbin]{\cup}{\cdot}}
\newcommand{\bigdcup}{\charfusion[\mathop]{\bigcup}{\cdot}}

\DeclareFontFamily{U}  {MnSymbolC}{}
\DeclareSymbolFont{MnSyC}         {U}  {MnSymbolC}{m}{n}
\DeclareFontShape{U}{MnSymbolC}{m}{n}{
    <-6>  MnSymbolC5
   <6-7>  MnSymbolC6
   <7-8>  MnSymbolC7
   <8-9>  MnSymbolC8
   <9-10> MnSymbolC9
  <10-12> MnSymbolC10
  <12->   MnSymbolC12}{}
\DeclareMathSymbol{\powerset}{\mathord}{MnSyC}{180}

\makeatletter
\def\namedlabel#1#2{\begingroup
    #2%
    \def\@currentlabel{#2}%
    \phantomsection\label{#1}\endgroup
}
\makeatother

\newtheorem{teorema}             {Teorema}
\newtheorem{lema}     	[teorema] {Lema}
\newtheorem{conjectura}	[teorema] {Conjectura}
\newtheorem{propriedade}[teorema] {Propriedade}
\newtheorem{definicao}	[teorema] {Definição}
\newtheorem{proposicao} [teorema] {Proposição}
\newtheorem{corolario}	[teorema] {Corolário}
\newtheorem{fato}	[teorema] {Fato}
\newtheorem{afirmacao}	[teorema] {Afirmação}

\newtheoremstyle{remark}  {2pt}  {4pt}  {\rm}  {}  {\bfseries}  {.}  {.3em}          {}
\theoremstyle{remark}
\newtheorem{observacao}	[teorema] {Observação}
\newtheorem{exemplo}	[teorema] {Exemplo}
\newtheorem{exercicio}	[teorema] {Exercicio}

%\renewcommand{\thefootnote}{\fnsymbol{footnote}}

\let\eps=\varepsilon
%\let\theta=\vartheta
\let\rho=\varrho
\let\phi=\varphi

\def\CC{\mathds C}
\def\NN{\mathds N}
\def\ZZ{\mathds Z}
\def\QQ{\mathds Q}
\def\RR{\mathds R}
\def\PP{\mathds P}
\def\EE{\mathds E}

\def\cB{\mathcal B}
\def\cF{\mathcal F}
\def\cL{\mathcal L} % para o arquivo 2018.03.19
\def\cR{\mathcal R}

\def\ra{\longrightarrow}
\def\red{\text{\rm red}}
\def\blue{\text{\rm blue}}
\def\R{\text{\rm red}}
\def\B{\text{\rm blue}}
\def\lf{\left\lfloor}
\def\rf{\right\rfloor}

\usepackage{datetime}

\usepackage{lineno}
\newcommand*\patchAmsMathEnvironmentForLineno[1]{%
\expandafter\let\csname old#1\expandafter\endcsname\csname #1\endcsname
\expandafter\let\csname oldend#1\expandafter\endcsname\csname end#1\endcsname
\renewenvironment{#1}%
{\linenomath\csname old#1\endcsname}%
{\csname oldend#1\endcsname\endlinenomath}}%
\newcommand*\patchBothAmsMathEnvironmentsForLineno[1]{%
\patchAmsMathEnvironmentForLineno{#1}%
\patchAmsMathEnvironmentForLineno{#1*}}%
\AtBeginDocument{%
\patchBothAmsMathEnvironmentsForLineno{equation}%
\patchBothAmsMathEnvironmentsForLineno{align}%
\patchBothAmsMathEnvironmentsForLineno{flalign}%
\patchBothAmsMathEnvironmentsForLineno{alignat}%
\patchBothAmsMathEnvironmentsForLineno{gather}%
\patchBothAmsMathEnvironmentsForLineno{multline}%
}

\mathcode`l="8000
\begingroup
\makeatletter
\lccode`\~=`\l
\DeclareMathSymbol{\lsb@l}{\mathalpha}{letters}{`l}
\lowercase{\gdef~{\ifnum\the\mathgroup=\m@ne \ell \else \lsb@l \fi}}%
\endgroup

