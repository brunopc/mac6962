% Time-stamp: <Sunday 08 Apr 2018 07:10:14pm -03 by yoshi on pokopon>
\documentclass[reqno, a4paper, 11pt]{amsart}
\pdfoutput=1

\usepackage[brazil]{babel}
\usepackage[T1]{fontenc}
\usepackage[utf8]{inputenc}
\usepackage{textcomp}

\usepackage{ae}
\usepackage{aecompl}
\usepackage{lmodern}

\makeatletter
\let\origsection=\section \def\section{\@ifstar{\origsection*}{\mysection}}
\def\mysection{\@startsection{section}{1}\z@{.7\linespacing\@plus\linespacing}{.5\linespacing}{\normalfont\scshape\centering\S}}
\makeatother

\allowdisplaybreaks

\usepackage{fullpage}
\usepackage{setspace}
\usepackage{fancyhdr}
\pagestyle{fancy}

\lhead{}
\chead{}
\rhead{}
%\lfoot{}
\cfoot{}
%\rfoot{\today, \currenttime}
\fancyfoot[RO,LE]{\thepage}
\fancyfoot[LO,RE]{{\footnotesize\today, \currenttime}}
\renewcommand{\headrulewidth}{0pt}
\renewcommand{\footrulewidth}{0pt}

\usepackage{amsmath,amssymb,amsthm}
\usepackage{mathabx}\changenotsign
\usepackage{mathrsfs}
\usepackage{bbm}
\usepackage{dsfont}
\usepackage[babel]{microtype}
\usepackage{xcolor}
\usepackage[backref]{hyperref}
\hypersetup{
  colorlinks,
  linkcolor={red!60!black},
  citecolor={green!60!black},
  urlcolor={blue!60!black},
}
\usepackage{algpseudocode}

\usepackage{bookmark}

%%%% Let's use amsrefs http://www.ams.org/publications/authors/tex/amsrefs
\usepackage[abbrev,msc-links,backrefs]{amsrefs} 
\usepackage{doi}
\renewcommand{\doitext}{DOI\,}
\renewcommand{\PrintDOI}[1]{\doi{#1}}
\renewcommand{\eprint}[1]{\href{http://arxiv.org/abs/#1}{arXiv:#1}}
%%%%

\usepackage{tikz}
\usetikzlibrary{decorations.pathreplacing,calc}
\pgfdeclarelayer{background}
\pgfdeclarelayer{foreground}
\pgfsetlayers{background,main,foreground}

\newcommand\inj\hookrightarrow % Injective functions
\DeclareMathOperator\Var{Var} % Variance
\DeclareMathOperator\Cov{Cov} % Covariance

\usepackage[abbrev,msc-links,backrefs]{amsrefs}
\usepackage{doi}
\renewcommand{\doitext}{DOI\,}
\renewcommand{\PrintDOI}[1]{\doi{#1}}
\renewcommand{\eprint}[1]{\href{http://arxiv.org/abs/#1}{arXiv:#1}}

%\usepackage[english]{babel}
%\numberwithin{equation}{section}

\makeatletter
\def\@setdate{\datename\ \@date}
%\def\@setthanks{\def\thanks##1{\par##1}\thankses}
\def\@setaddresses{\par
  \nobreak \begingroup
\footnotesize
  \def\author##1{\nobreak\addvspace\bigskipamount}%
  \def\\{\unskip, \ignorespaces}%
  \interlinepenalty\@M
  \def\address##1##2{\begingroup
    \par\addvspace\bigskipamount\indent
    \@ifnotempty{##1}{(\ignorespaces##1\unskip) }%
    {\scshape\ignorespaces##2}\par\endgroup}%
  \def\curraddr##1##2{\begingroup
    \@ifnotempty{##2}{\nobreak\indent{\itshape Current address}%
      \@ifnotempty{##1}{, \ignorespaces##1\unskip}\/:\space
      ##2\par}\endgroup}%
  \def\email##1##2{\begingroup
    \@ifnotempty{##2}{\nobreak\indent{\itshape Endereço eletrônico}%
      \@ifnotempty{##1}{, \ignorespaces##1\unskip}\/:\space
      \ttfamily##2\par}\endgroup}%
  \def\urladdr##1##2{\begingroup
    \@ifnotempty{##2}{\nobreak\indent{\itshape URL}%
      \@ifnotempty{##1}{, \ignorespaces##1\unskip}\/:\space
      \ttfamily##2\par}\endgroup}%
  \addresses
  \endgroup
}
\def\datename{\textit{Data}:}
\makeatother

\usepackage{enumitem}
\def\rmlabel{\upshape({\itshape \roman*\,})}
\def\RMlabel{\upshape(\Roman*)}
\def\alabel{\upshape({\itshape \alph*\,})}
\def\Alabel{\upshape({\itshape \Alph*\,})}
\def\nlabel{\upshape({\itshape \arabic*\,})}
\def\AAlabel{\upshape({A\arabic*})}
\def\nplain{\upshape{\arabic*.}}
\def\aplabel{\upshape({\itshape \alph*\,$'$})}

\let\polishlcross=\l
\def\l{\ifmmode\ell\else\polishlcross\fi}

\def\tand{\ \text{and}\ }
\def\qand{\quad\text{and}\quad}
\def\qqand{\qquad\text{and}\qquad}

\let\emptyset=\varnothing
\let\setminus=\smallsetminus
\let\backslash=\smallsetminus
\let\subset\subseteq
\let\lg = \log
\let\log=\ln

\makeatletter
\def\moverlay{\mathpalette\mov@rlay}
\def\mov@rlay#1#2{\leavevmode\vtop{   \baselineskip\z@skip \lineskiplimit-\maxdimen
   \ialign{\hfil$\m@th#1##$\hfil\cr#2\crcr}}}
\newcommand{\charfusion}[3][\mathord]{
    #1{\ifx#1\mathop\vphantom{#2}\fi
        \mathpalette\mov@rlay{#2\cr#3}
      }
    \ifx#1\mathop\expandafter\displaylimits\fi}
\makeatother

\DeclareMathOperator{\dom}{{\rm dom}}

\newcommand{\dcup}{\charfusion[\mathbin]{\cup}{\cdot}}
\newcommand{\bigdcup}{\charfusion[\mathop]{\bigcup}{\cdot}}

\DeclareFontFamily{U}  {MnSymbolC}{}
\DeclareSymbolFont{MnSyC}         {U}  {MnSymbolC}{m}{n}
\DeclareFontShape{U}{MnSymbolC}{m}{n}{
    <-6>  MnSymbolC5
   <6-7>  MnSymbolC6
   <7-8>  MnSymbolC7
   <8-9>  MnSymbolC8
   <9-10> MnSymbolC9
  <10-12> MnSymbolC10
  <12->   MnSymbolC12}{}
\DeclareMathSymbol{\powerset}{\mathord}{MnSyC}{180}

\makeatletter
\def\namedlabel#1#2{\begingroup
    #2%
    \def\@currentlabel{#2}%
    \phantomsection\label{#1}\endgroup
}
\makeatother

\newtheorem{teorema}             {Teorema}
\newtheorem{lema}     	[teorema] {Lema}
\newtheorem{conjectura}	[teorema] {Conjectura}
\newtheorem{propriedade}[teorema] {Propriedade}
\newtheorem{definicao}	[teorema] {Definição}
\newtheorem{proposicao} [teorema] {Proposição}
\newtheorem{corolario}	[teorema] {Corolário}
\newtheorem{fato}	[teorema] {Fato}
\newtheorem{afirmacao}	[teorema] {Afirmação}

\newtheoremstyle{remark}  {2pt}  {4pt}  {\rm}  {}  {\bfseries}  {.}  {.3em}          {}
\theoremstyle{remark}
\newtheorem{observacao}	[teorema] {Observação}
\newtheorem{exemplo}	[teorema] {Exemplo}
\newtheorem{exercicio}	[teorema] {Exercicio}

%\renewcommand{\thefootnote}{\fnsymbol{footnote}}

\let\eps=\varepsilon
%\let\theta=\vartheta
\let\rho=\varrho
\let\phi=\varphi

\def\CC{\mathds C}
\def\NN{\mathds N}
\def\ZZ{\mathds Z}
\def\QQ{\mathds Q}
\def\RR{\mathds R}
\def\PP{\mathds P}
\def\EE{\mathds E}

\def\cB{\mathcal B}
\def\cF{\mathcal F}
\def\cL{\mathcal L} % para o arquivo 2018.03.19
\def\cR{\mathcal R}

\def\ra{\longrightarrow}
\def\red{\text{\rm red}}
\def\blue{\text{\rm blue}}
\def\R{\text{\rm red}}
\def\B{\text{\rm blue}}
\def\lf{\left\lfloor}
\def\rf{\right\rfloor}

\usepackage{datetime}

\usepackage{lineno}
\newcommand*\patchAmsMathEnvironmentForLineno[1]{%
\expandafter\let\csname old#1\expandafter\endcsname\csname #1\endcsname
\expandafter\let\csname oldend#1\expandafter\endcsname\csname end#1\endcsname
\renewenvironment{#1}%
{\linenomath\csname old#1\endcsname}%
{\csname oldend#1\endcsname\endlinenomath}}%
\newcommand*\patchBothAmsMathEnvironmentsForLineno[1]{%
\patchAmsMathEnvironmentForLineno{#1}%
\patchAmsMathEnvironmentForLineno{#1*}}%
\AtBeginDocument{%
\patchBothAmsMathEnvironmentsForLineno{equation}%
\patchBothAmsMathEnvironmentsForLineno{align}%
\patchBothAmsMathEnvironmentsForLineno{flalign}%
\patchBothAmsMathEnvironmentsForLineno{alignat}%
\patchBothAmsMathEnvironmentsForLineno{gather}%
\patchBothAmsMathEnvironmentsForLineno{multline}%
}

\mathcode`l="8000
\begingroup
\makeatletter
\lccode`\~=`\l
\DeclareMathSymbol{\lsb@l}{\mathalpha}{letters}{`l}
\lowercase{\gdef~{\ifnum\the\mathgroup=\m@ne \ell \else \lsb@l \fi}}%
\endgroup

\begin{document}
\linenumbers

\title[MAC6962 Tópicos Matemáticos para Computação Contemporânea]{%
{\small\sl Sinopse das aulas}\\\bigskip
MAC6962 Tópicos Matemáticos para Computação Contemporânea\\\bigskip
{\it Segundo Semestre de 2019}
}

\address{Instituto de Matemática e Estatística, Universidade de São
  Paulo, Rua do Matão 1010, 05508--090~São Paulo, SP}

% \email{}
%\thanks{}
%\begin{abstract}
%\end{abstract}

\yyyymmdddate
\shortdate
\def\today{\number\year/\number\month/\number\day}
\settimeformat{ampmtime}
\date{\today, \currenttime}
\footskip=28pt

%\keywords{}
%\subjclass[2010]{}

\maketitle

\section*{}
\label{sec:prefacio}
\doublespace
Notas de aula produzidas por
\begin{enumerate}[label=\nplain]
\item \dots,
\item \dots,
\item \dots\ e
\item \dots.
\end{enumerate}

\endgroup
\newpage\onehalfspace
\tableofcontents
\pagestyle{fancy}

\endgroup
\doublespace
%\onehalfspace

\newpage
\part{CONCENTRAÇÃO DE MEDIDA}

\section{...}
\label{...}

\section{Aula 05 de Agosto de 2019}
\label{2019_08_05}

\subsection{Filtros de Bloom}

Considere o seguinte problema: dado um $S \subseteq \mathcal{U}$ e $x
\in \mathcal{U}$, decidir se $x \in S$. A solução para esse problema
consiste em montar alguma Estrutura de Dados para representar $S$,
como por exemplo uma tabela de hash ou uma árvore binária de busca
(ABB), e fazer consultas nessa estrutura. Queremos que a nossa solução
seja eficiente em tempo e espaço, sendo que o tempo consiste no tempo
de montar a estrutura (pré-processamento) e no tempo de resposta para
cada $x$ (consulta).

Na Tabela \ref{tab:tempo_problema_pertinencia}, temos o tempo de
pré-processamento e consulta utilizando hashing e ABB. Observe que nas
duas soluções a quantidade de espaço utilizado é $O(n)$. Queremos
gastar menos memória, para isso vamos permitir aleatorização e falsos
positivos.

\begin{table}[h]
\begin{tabular}{l|c|c|}
\cline{2-3}
                                       & \multicolumn{2}{c|}{\textbf{Tempo}}                                                      \\ \cline{2-3} 
                                       & \multicolumn{1}{l|}{\textbf{Pré-processamento}} & \multicolumn{1}{l|}{\textbf{Consulta}} \\ \hline
\multicolumn{1}{|l|}{\textbf{Hashing}} & $O(n)$                                          & $O(1)$                                 \\ \hline
\multicolumn{1}{|l|}{\textbf{ABB}}     & $O(n\lg n)$                                    & $O(\lg n)$                            \\ \hline
\end{tabular}
  \caption{Tempo utilizando Hashing e ABB}
  \label{tab:tempo_problema_pertinencia}
\end{table}

Mais precisamente, teremos:

\begin{enumerate}
\item Se $x \in S$, $\mathbb{P}(A(x)=1)=1$
\item Se $x \notin S$, $\mathbb{P}(A(x)=1) \leq \delta$.
\end{enumerate}

onde $A(x)$ é a solução desejada com $x$ como entrada.

\subsubsection{Primeira Tentativa}
  
Suponha que temos uma função de hashing
$h: \mathcal{U} \rightarrow T$, onde $|T|=m$. Vamos considerar que $h$
tem a seguinte propriedade: $h(u)$, para todo $u \in \mathcal{U}$, são
uniformemente distribuído em $T$ $\left(
\mathbb{P}(h(u)=t)=\frac{1}{m}\right)$ e são mutuamente independentes (essa
propriedade é usual em análises).

Vamos representar o conjunto $S$ utilizando um vetor $v$ de $m$
bits. Isto é:

\begin{align*}
v = (v_{t})_{t \in T} = (v(t))_{t\in T} \in \{0,1\}^{m}
\end{align*}

com

\begin{align*}
  v_{t}=\begin{cases}
    1, \text{ se } \exists s \in S : h(s)=t \\
    0, \text{ c/c. }
 \end{cases}
\end{align*}

Neste caso, o algoritmo $A$ devolverá $1$ se $v(h(x))=1$ e $0$ caso
contrário. Logo, $A(x)=v(h(x))$. Observe que se $x \in S$, então
$A(x)=1$ com probabilidade $1$ e se $x \notin S$, então
$\mathbb{P}(A(x)=1)\leq \frac{|S|}{|T|}$.

Consequentemente, para termos a probabilidade de falsos positivos
menor que $\delta$, basta tomarmos $|T| \geq
\frac{|S|}{\delta}$. Portanto, gastamos $\frac{1}{\delta}$ bits por
elemento de $S$. Queremos diminuir $\frac{1}{\delta}$ para
$O(\lg \frac{1}{\delta})$.

\subsubsection{Filtros de Bloom}

Neste caso, usamos $k$ funções de hashing
$h_{1},\dots,h_{k}:\mathcal{U} \rightarrow T$ e vamos assumir que
$h_{i}(u) (1 \leq i \leq k, u \in \mathcal{U})$ são uniformemente
distribuídas em $T$ e são mutuamente independentes. Como na tentativa
anterior, ainda vamos representar $S$ utilizando um vetor $v$ de $m$
bits, mas dessa vez:

\begin{align*}
  v_{t}=\begin{cases}
    1, \text{ se } \exists i \text{ e } s \in S : h_{i}(s)=t \\
    0, \text{ c/c. }
  \end{cases}
\end{align*}

Ademais:

\begin{align*}
  A(x)=\begin{cases}
    1, \text{ se } \forall i v(h_{i}(x))=1\\
    0, \text{c/c.}
  \end{cases}
\end{align*}

Claramente, se $x \in S$, então $A(x)=1$ com probabilidade 1. Queremos
estimar $\mathbb{P}(A(x)=1)$ para $x \notin S$. Lembre que $n=|S|$ e
$m=|T|$. Tomamos $m=n\frac{\lg_{2} 1/\eps}{\log 2}$ e $k=(\log 2)\frac{m}{n}$,
onde vamos escolher $\eps$ mais a frente \footnote{quando não crucial,
omitimos $\lfloor . \rfloor$ e $\lceil . \rceil$. Além disso,
$\lg=\lg_{e}=\log$}.

Como é $v=(v_{t})$? Temos:

\begin{align}
\mathbb{P}(v_{t}=0) &= \left(1-\frac{1}{m}\right)^{kn}\\
                    &= \left(e^{-\frac{1}{m}+O\left(\frac{1}{m^2}\right)}\right)^{kn} \label{eq:pr_vt_0}, & \left(\text{usando que } 1+x=e^{x+O(\frac{1}{x^2})}\right) \\
                    &= \frac{1}{2}e^{O\left(\frac{1}{m}\right)} \geq \frac{m}{2}\left(1-\frac{e}{2}\right), & (m \geq m_{0}(\eps))
\end{align}

Observe que em \eqref{eq:pr_vt_0}, assumimos que $O$ não tem
sinal. Iremos assumir isso no restante das aulas. Agora, seja $M_{0}$
uma variável aleatória tal que $M_{0}=\#0s$ em $v$. Então:

$$
\mathbb{E}(M_{0})=m\mathbb{P}(v_{t}=0) \geq \frac{m}{2}\left(1-\frac{\eps}{2}\right)
$$

\begin{fato}
\label{fato:filtros_de_Bloom}
$\mathbb{P}(M_{0} \leq \frac{m}{2}(1-\eps)) \leq e^{-c_{\eps}m}$, onde $c_{\eps} > 0$ depende apenas de $\eps$.
\end{fato}

Dado o fato acima, supomos que $M_{0} \geq \frac{m}{2}(1-\eps)$. Logo,
$M_{1}=m-M_{0} \leq \frac{m}{2}(1+\eps)$. Assim, se $x \notin S$, então:

\begin{align*}
  \mathbb{P}(A(x)=1)&= \left(\frac{M_{1}}{m}\right)^k \\
                    &\leq \left(\frac{1}{2}(1+\eps)\right)^k \\
                    &= \eps(1+\eps)^{\lg_{2}1/\eps}, & \left(k=\lg_{2}1/\eps\right)\\
                    &\leq \eps e^{\eps\lg_{2}1/\eps}, & \left(\text{usando que } (1+x) \leq e^{x}\right)\\
                    &= \eps e^{\lg\eps^{-\eps}/\lg 2}\\
                    &= \eps \eps^{-\eps/\lg 2}\\
                    &\leq 2 \eps, & \left(\text{para algum } 0 \leq \eps_{0} \leq \eps\right)\\
                    &= \delta
\end{align*}

tomando $\eps=\frac{\delta}{2}$. Portanto,
$m=n\frac{\lg 2/\delta}{\lg 2}$, ou seja, gastamos
$O(\lg \frac{1}{\delta})$ para cada elemento de $S$.

Agora vamos provar o Fato \ref{fato:filtros_de_Bloom}. Para isso,
vamos usar um resultado de concentração de variáveis aleatórias
conhecida como a desigualdade de McDiamird ou o método das diferenças
limitadas:

\begin{lema}[McDiamird \cite{mcdiarmid_method_89}]
\label{lema:McDiamird}
Suponha que $X_{1},\dots,X_{n}$ são variáveis aleatórias
independentes, com $X_{i}$ tomando valores em um conjunto $A_{i}(1
\leq i \leq n)$. Suponha que $f: A_{1}\times \dots A_{n} \rightarrow
\RR$ é tal que $|f(x)-f(x')|\leq c_{i}$ para quaisquer $x$ e $x' \in
A_{1}\times \dots \times A_{n}$ que diferem apenas na $i$-ésima
coordenada. Considere a variável aleatória
$Y=f(X_{1},\dots,X_{n})$.Para todo $t > 0$:

  \begin{enumerate}
    \item $\mathbb{P}(Y \geq \mathbb{E}(Y_0)+t) \leq e^{\frac{-2t^{2}}{\sum c_{i}^{2}}}$
    \item $\mathbb{P}(Y \geq \mathbb{E}(Y_0)-t) \leq e^{\frac{-2t^{2}}{\sum c_{i}^{2}}}$
    \end{enumerate}
\end{lema}

Portanto:

\begin{proof}[Demonstração do Fato \ref{fato:filtros_de_Bloom}]
Tomando $t = \frac{\eps m}{4}$, temos:

\begin{align*}
  \mathbb{E}(M_{0}) - t &\geq \frac{m}{2}\left(1-\frac{\eps}{2}\right) - t\\
                        &= \frac{m}{2}\left(1-\frac{\eps}{2}\right) - \frac{\eps m}{4}\\
                        &= \frac{m}{2}\left(1-\eps\right)
\end{align*}

Logo, usando o Lema \ref{lema:McDiamird}:

\begin{align*}
  \mathbb{P}\left(M_{0}\leq \frac{m}{2}\left(1-\eps\right)\right) &\leq \mathbb{P}\left(M_{0}\leq \mathbb{E}(M_{0})-t\right)\\
                                                                  &\leq e^{-2t^2/kn}\\
                                                                  &\leq e^{-c_{\eps}m}
\end{align*}

onde $c_{\eps}$ é uma constante que depende somente de $\eps$.

\end{proof}


\newpage
\part{BIBLIOGRAFIA}

%\bibliographystyle{amsplain}
\bibliography{bibliography}

\endgroup
\end{document}

%%% Local Variables:
%%% mode: latex
%%% eval: (auto-fill-mode t)
%%% eval: (LaTeX-math-mode t)
%%% eval: (flyspell-mode t)
%%% TeX-master: t
%%% End:
