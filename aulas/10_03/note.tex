\section{Aula 03 de Outubro de 2019}
\label{2019_10_03}

\subsection{Redução de dimensão por projeções aleatórias}
Nessa seção iremos discutir o problema de reduzir a dimensão de um conjunto de pontos por meio de projeções com componentes aleatórias preservando alguma característica do conjunto de pontos.
Sejam $p_1,...p_n \in \mathbb{R}^d$ pontos e seja $A_{n \times d}$ a matriz cuja as linhas são esses pontos, temos que, usando decomposição SVD:
\begin{align*}
A &=
  \begin{bmatrix}
    P_1^T \\
    \vdots \\
    P_n^T
  \end{bmatrix}\\
   A &= \sum\limits_{i = 1}^r \sigma_i u_i^T v_i    
\end{align*}

\paragraph{}com $posto(A) = r$. Seja $A_k$ em que: 
\begin{equation*}
    A_k = \sum\limits_{i = 1}^k \sigma_i u_i^T v_i    
\end{equation*}
\paragraph{}e para facilitar suponha o centroide na origem. Temos que, para toda matriz $D$ tal que $ posto(D)=k$:
\begin{equation*}
    \left\Vert A - A_k \right\Vert_F \leq \left\Vert A - D \right\Vert_F
\end{equation*}
em que $\left\Vert \right\Vert_F$ é a norma de Frobenius.
\paragraph{}A redução por SDV não preserva as distâncias, a mesma minimiza a soma total dos deslocamentos ao quadrado. Vamos ver outra forma de reduzir a dimensão em que preservamos um alta probabilidade um característica do conjunto de pontos.
\subsubsection{Projeções que preservam distância} Há aplicações em que é útil preservar as distâncias entre os pontos. Suponha o problema em que é dado um conjunto $P$ de $n$ pontos em $\mathbb{R}^d$, com $n$ e $d$ grandes, e são feitas várias queries. Em cada querie recebemos um ponto $q \in \mathbb{R}^d$ e o objetivo é descobrir o ponto de $P$ mais próximo de $q$. Para obter um algoritmo eficiente podemos usar aleatorização e permitir soluções aproximadas. Vamos considerar um algoritmo bom se caso a resposta para um querie seja $p_i \in P$ garantidamente não há $p_j \in P$ tal que $\left\Vert p_j - q \right\Vert <(1 - \epsilon) \left\Vert p_i - q \right\Vert$. 

\paragraph{}Considere a aplicação linear $R \colon \mathbb{R}^d \to \mathbb{R}^k$ aleatória, definida da seguinte forma: Escolha $k$ vetores gaussianos $u_1,...,u_k \sim N_d(0,1)$ independentes. Para $v \in \mathbb{R}^d$, pomos:
\begin{equation*}
    \begin{split}
    R(v) &= (u_1^Tv,...,u_k^Tv)\\
         &= \sum\limits_{i=1}^k (u_i^Tv)e_i
    \end{split}
\end{equation*}
\paragraph{}Vamos provar que se definirmos $T = \frac{1}{\sqrt{k-1}}R$, então para $v \in \mathbb{R}^d fixo$:
\begin{equation*}
    \left\Vert T(v) \right\Vert \sim \left\Vert v \right\Vert
\end{equation*}
\paragraph{}com alta probabilidade. De fato, se $x_1,...,x_n\in \mathbb{R}^d$, temos que:
\begin{equation*}
    \left\Vert T(x_i) - T(x_j) \right\Vert \sim \left\Vert x_i - x_j \right\Vert
\end{equation*}
\paragraph{}Para todo $i,j$ desde que $k$ seja grande o suficiente em relação a n.
\begin{fato}
Sejam $u_1,...,u_k \sim N_d(0,1)$ independentes e seja $v \in \mathbb{R}^d$ com $\left\Vert V \right\Vert = 1$ temos que $u_i^Tv \sim N(0,1)$.
\end{fato}
\paragraph{}Como esses vetores $u_1,...,u_k \sim N_d(0,1)$ são independentes e $v = (v_1,...,v_d)$, temos que:
\begin{align*}
    u_{i1},...,u_{id} &\sim N(0,1), \text{ uma normal em cada coordenada}\\
    u^T_iv &= u_{i1}v_1 +...+u_{id}v_d
\end{align*}
\paragraph{}Segue do fato que se $v \in \mathbb{R}^d$ e $\left\Vert V \right\Vert = 1$, então $R(v) \sim N_k(0,1)$. Pelo teorema do anel (Gaussian Annulus Theorem).
\begin{equation*}
    \mathbb{P}\left(\left\vert \left\Vert R(v) \right\Vert - \sqrt{k-1} \right\vert > c\right) \leq \frac{4}{c^2}e^{-\frac{c^2}{4}}
\end{equation*}
\paragraph{} Tome $c = \epsilon\sqrt{k-1}$. Então
\begin{equation*}
    \mathbb{P}(\left\vert \left\Vert R(v) \right\Vert - \sqrt{k-1} \right\vert > \epsilon\sqrt{k-1}) \leq \frac{4}{\epsilon^2(k-1)}e^{-\frac{(k-1)\epsilon^2}{4}}
\end{equation*}
\paragraph{}Tome $T=\frac{1}{\sqrt{k-1}}R$. Então:
\begin{equation*}
    \mathbb{P}(\left\vert \left\Vert T(v) \right\Vert - \left\Vert v \right\Vert \right\vert > \epsilon\left\Vert v \right\Vert) \leq \frac{4}{\epsilon^2(k-1)}e^{-\frac{(k-1)\epsilon^2}{4}}
\end{equation*}
\paragraph{}Suponha agora que temos pontos $x_1,..,x_n \in \mathbb{R}^d$ fixos. Tome $k \geq 12\epsilon^{-2}\log{n}$ e suponha $n>n_0$ para que:
\begin{equation*}
    \epsilon^2(k-1) \geq 4
\end{equation*}
\paragraph{}Note que $(k-1)\frac{\epsilon^2}{4} \geq 3\log(n)$. Assim, para $i \neq j$ fixos, temos:
\begin{equation*}
    \mathbb{P}(\left\vert \left\Vert T(x_i) - T(x_j) \right\Vert - \left\Vert x_i - x_j \right\Vert \right\vert > \epsilon\left\Vert x_i - x_j \right\Vert) \leq \frac{1}{n^3}
\end{equation*}
\paragraph{}Assim, seja $B_{ij} = \left\vert \left\Vert T(x_i) - T(x_j) \right\Vert - \left\Vert x_i - x_j \right\Vert \right\vert > \epsilon\left\Vert x_i - x_j \right\Vert$
\begin{equation*}
    \mathbb{P}(\exists\> i \neq j : B_{ij}) \leq \binom{n}{2}\frac{1}{n^3} \leq \frac{1}{2n}
\end{equation*}
\paragraph{}Portanto, provamos a seguinte versão do lema de Johnson e Lidenstrauss:
\begin{teorema}[Johnson e Lidenstrauss]
Seja $T \colon \mathbb{R}^d \to \mathbb{R}^k$ como definido acima com $k > 12\epsilon^{-2}\log{n}$. Sejam $x_1,...,x_n \in \mathbb{R}^d (n \geq n_0)$ dados. Com probabilidade maior que $1 - \frac{1}{2n}$, temos que, para todo $i,j$:
\begin{equation*}
    \left\Vert T(x_i) - T(x_j) \right\Vert   = (1 \pm \epsilon)\left\Vert x_i - x_j \right\Vert
\end{equation*}
\end{teorema}

\paragraph{}Algumas observações:
\begin{enumerate}
    \item É possível provar com resultado análogo com $T$ a projeção ortogonal a um subespaço $k$-dimensional aleatório.
    \item Podemos usar distribuições mais simples do que $N_d(0,1)$ para os $u_i(1 \leq i \leq k)$. Suponha $u_i = (u_{i1,...,id})$. Podemos usar as seguintes duas distribuições:
    \begin{enumerate}
        \item 
        \[
                 u_{ij}=   \begin{cases} 
                        1,&\ \mathbb{P}=\frac{1}{2}\\ 
                        -1,&\ \text{c.c}\end{cases}
        \]
        \item
        \[
                 u_{ij}=   \begin{cases} 
                        \sqrt{3},&\ \mathbb{P}=\frac{1}{6}\\ 
                        0,&\ \mathbb{P}=\frac{2}{3} \\
                        -\sqrt{3},&\ \text{c.c}\end{cases}
        \]
    \end{enumerate}
\end{enumerate}
\paragraph{} Ambas as distribuições são mais simples que usar binomiais e apresentam resultados de qualidade equivalente $k = \Omega(\epsilon^{-2}\log{n})$. A distribuição (b) pode gerar uma matriz esparsa. Uma análise mais profunda pode ser encontrada em \cite{achlioptas_2003}
e \cite{bingham_2001}.
