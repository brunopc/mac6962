\section{Aula 15 de Outubro de 2019}
\label{2019_10_15}

\subsection{Grafos aleatórios}

A partir dessa aula estudaremos teoria básica de grafos aleatórios e uma aplicação desses ao problema de \emph{heshing} - \emph{cucleoo heshing}.

O foco seŕa estudar propriedades $\mathcal{P}$ (propriedades invariantes por isomorfismo) de grafos de $n$ vértices - $G = G^n$.

Queremos estudar $\lim_{n \rightarrow \infty}\mathbb{(G^n \in \mathcal{P})}$. Em geral, os resultados dizem que esse limite é 0 ou 1.

\begin{definicao}
$\mathcal{P}$ é crescente se $G \in \mathcal{P}$ então $G + e \in \mathcal{P}$, para todo $c \in \binom{V(G)}{2}$.
\end{definicao}{}

\begin{exemplo}
"Ser connexo" é ser crescente.
\end{exemplo}{}

\subsubsection{Modelos - Erdos \& Rényi}
\begin{enumerate}
    \item $G(n, p)$ ou mais compactamente $G_{n,p}$, indica $n$ vértices com $0\leq p \leq 1$, sendo tipicamente $p = p(n)$.
\end{enumerate}

Neste caso, $V(G_p)$ é conjunto fixo de $n$ vértices, com cada $\{x, y\} \in \binom{V(G)}{2}$ pertence a $G_p$ com probabilidade $p$ (Todos independentes).

\begin{enumerate}    
\setcounter{enumi}{1}
    \item $G(n, M)$ - $G_M$ vértices, com $(V = V(G(n, M)))$ o conjunto fixo de $n$ vértices.
\end{enumerate}

Neste modelo, os $\binom{\binom{V}{2}}{M}$ grafos $H$ com $V(H) = 1$ e $|E(H)| = M$ são equiprováveis.
\begin{equation*}
    \mathbb{P}(G(n, M) = H) = \binom{\binom{n}{2}}{M}^{-1}
\end{equation*}{}

Observe que $G(n,M)$ pode ser estudado considerando-se $G(n,p)$ com $p = M / \binom{n}{2}$.


\begin{enumerate}    
\setcounter{enumi}{2}
    \item $G = (G_t)_{t = 0}^N$, sendo $N = \binom{n}{2}$ e além disso $\phi = G_0 \subset G_1 \subset \dots \subset G_N = k^N$
\end{enumerate}

$G_t$ é obtido de $G_{t-1}$ adicionando-se uma aresta nova aleatória.

O espaço $\Omega = \Omega_n$ desse G tem $N! = \binom{n}{2}!$ elementos equiprováveis. Ademais, $G_t$ tem $t$ arestas e a distribuição de $G_t$ é $G(n, t)$ (2° modelos).

\begin{exemplo}
Seja $\mathcal{P}$ uma propriedade crescente tal que $k^N \in \mathcal{P}$, então $T_p(G) = \min(t:\mathcal{P} \subset G_t )$.

Considere as seguintes propriedades:
\begin{enumerate}
    \item $\mathcal{P} = \{G \text{é concexo}\}$;
    \item $\mathcal{P}' = \{\delta \geq 1\} = \{G: \delta(G) \geq 1\} = G$ não tem vértice isolado.
\end{enumerate}{}

Logo $\mathcal{P} \subset \mathcal{P}'$. Ademais $T_{con}(G) \geq T_{\delta \geq 1}(G)$.
\end{exemplo}
\begin{teorema}
Assintoticamente quase todo grafo $G = (G_t)_0^N$ é tal que $T_{con}(G) = T_{\delta \geq 1}(G)$
\end{teorema}{}

\begin{enumerate}
\setcounter{enumi}{3}
    \item Outros modelos - "\emph{Network science}"
\end{enumerate}{}

\subsubsection{Teoremas de Monotonicidade}
\begin{lema}
Suponha $0 \leq p \leq p' \leq 1$ e seja $\mathcal{P}$ uma propriedade crescente, então $\mathbb{P}(G(n,p) \in \mathcal{P} \leq \mathbb{P}(G(n,p') \in \mathcal{P})$
\end{lema}{}
\begin{proof}
\emph{Exposição em duas rodadas}

Sem perda de generalidade, suponha $p \leq p'$, e seja $kp{''} < 1$, tal que $1-p' = (1-p)(1 - p{''})$. Consideramos $G(n, p') = G(n, p) \cup G(n, p{''})$.

Como $p'$ é crescente tem-se $\mathbb{P}(G(n,p) \in \mathcal{P} \leq \mathbb{P}(G(n,p') \in \mathcal{P})$.
\end{proof}{}

Observe que resultado análogo vale para $G(n,M)$.
\begin{exercicio}
Enuncie e prove esse resultado.
\end{exercicio}{}

\subsubsection{Teoremas de Equivalência}
\begin{lema}
Sejam $\mathcal{P}$ uma propriedade crescente, $p = M/ N$, $M = M(n) \rightarrow \infty$ e $\delta > 0$ uma constante tal que 
\begin{equation*}
    (1 + \delta)\frac{M}{N} = (1 + \delta)M\binom{n}{2}^{-1} \leq 1
\end{equation*}{}
então valem as seguintes afirmações:
\begin{enumerate}
    \item $\mathbb{P}(G(n,p) \in \mathcal{P}) \rightarrow 1 \Rightarrow \mathbb{P}(G(n, M) \in \mathcal{P}) \rightarrow 1$;
    \item $\mathbb{P}(G(n,p) \in \mathcal{P}) \rightarrow 0 \Rightarrow \mathbb{P}(G(n, M) \in \mathcal{P}) \rightarrow 0$;
    \item $\mathbb{P}(G(n,M) \in \mathcal{P}) \rightarrow 1 \Rightarrow \mathbb{P}(G(n, (1 + \delta)p) \in \mathcal{P}) \rightarrow 1$;
    \item $\mathbb{P}(G(n,p) \in \mathcal{P}) \rightarrow 0 \Rightarrow \mathbb{P}(G(n, (1 + \delta)p) \in \mathcal{P}) \rightarrow 0$.
\end{enumerate}
\end{lema}{}
\subsubsection{Funções Limiares}

\begin{definicao}
Se $p_0 = p_0(n)$ é uma função limiar para $\mathcal{P}$ se
\begin{equation*}
    \lim_{n \rightarrow \infty} \mathbb{P}(G(n, p) \in \mathcal{P}) = \begin{cases}
    0, \text{se } p \ll p_0 \\
    1, \text{se } p \gg p_0
    \end{cases}    
\end{equation*}
onde

\begin{equation*}
    p \ll p_0 : \lim_{n \rightarrow \infty} \frac{p(n)}{p_0(n)} = 0
\end{equation*}
\end{definicao}

A afirmação $p \ll p_0$, então $\lim_{n \rightarrow \infty} \mathbb{P}(G(n, p) \in \mathcal{P}) = 0$ é conhecida como 0-afirmação. A afirmação $p \gg p_0$, então $\lim_{n \rightarrow \infty} \mathbb{P}(G(n, p) \in \mathcal{P}) = 1$ é conhecida como 1-afirmação.

\begin{definicao}
Se $p_0 = p_0(n)$ é tal que, para todo $\epsilon > 0$ temos
\begin{equation*}
    \lim_{n \rightarrow \infty} \mathbb{P}(G(n, p) \in \mathcal{P}) = \begin{cases}
    0, \text{se } p \ll p_0 \\
    1, \text{se } p \gg p_0
    \end{cases}    
\end{equation*}
então dizemos que $p_0$ é uma função limiar severa (\emph{sharp}).
\end{definicao}


\begin{definicao}
Uma função limiar é frouxa (\emph{coarse}) se ela não é severa.
\end{definicao}

\begin{exemplo}
Se $\mathcal{P} = \{k ^4 \subset G(n,p)\}$, então $p_0 = n^{\frac{2}{3}}$ é função limiar (frouxa) para $\mathcal{P}$.
\end{exemplo}

\begin{exemplo}
Se $\mathcal{P} = \{\text{conexidade}\}$, então $p_0 = \frac{\log(n)}{n}$ é função limiar (severa) para $\mathcal{P}$.
\end{exemplo}

De fato vale a seguinte propriedade (Teorema \ref{teo: 5_2019_10_15}).
\begin{teorema}
\label{teo: 5_2019_10_15}
Suponha que $p = p(n) = \frac{1}{n}(\log(n) + c_n)$, então:
\begin{equation*}
    \lim_{n \rightarrow \infty} \mathbb{P}(G(n, p) \text{é conexo}) = \begin{cases}
    0, \text{se } c_n \rightarrow - \infty \\
    \exp(-\exp(-c)), \text{se } c_n \rightarrow c \in \mathbb{R} \\
    1, \text{se } c_n \rightarrow 1
    \end{cases}    
\end{equation*}
\end{teorema}

\emph{Preliminares para demonstração do Teorema \ref{teo: 5_2019_10_15}}

Vamos considerar apenas os casos $c_n \rightarrow -\infty$ e $c_n \rightarrow \infty$, e apenas discutiremos a intuição para o caso $c_n \rightarrow c \in \mathbb{R}$. Agora considere que:
\begin{enumerate}
    \item $10 \leq |c_n| \leq \log(\log(n))$;
    \item n é suficiente grande para que as desigualdades abaixo sejam verdadeiras;
    \item $X_k = X_k(G(n,p)) =$ \# componentes conexas, em particular $X_1 = $ \# vértices isolados em $G(n,p)$.
\end{enumerate}{}

\begin{proof}

\emph{Caso $c_n \rightarrow -\infty$}

Seja $\mu = \mathbb{E}(X_1)$. Temos 
\begin{align*}
    \mu & = n(1 -p)^{n - 1} = n(\exp(-p + O(p^2)))^n / (1 - p) = \\
        & = (1 - v(1))n \exp(-np + O(np^2)) = (1 - v(1))n \exp(\log(n) + c_n) = \\
        & = (1 + O(1))\exp(-c_n) \rightarrow \infty
\end{align*}{}

O número esperado de pares ordenados de vértices isolados é
\begin{align*}
    \mathbb{E}(X_1(X_1 - 1)) & = n(n-1)(1-p) \leq n^2 (1-p)^{2n -n}/ (1-p) = \\
                             & = u^2/(1 - p) \leq u^2 (1 + 2p)
\end{align*}
Assim

\begin{equation*}
    \mathbb{E}(X_1) = \mathbb{E}(X_1(X_1 - 1)) + \mathbb{E}(X_1) \leq u^2 (1 + 2p) + \mu
\end{equation*}
Assim
\begin{equation*}
    \mathbb{E}((X_1 - \mu)^2) = \mathbb{E}(X_1^2) - u^2 \leq 2pu^2 + \mu
\end{equation*}
e além disso

\begin{align*}
    \mathbb{P}(G(n,p) \text{ é conexo}) & \leq \mathbb{P}(X_1 = 0) \leq \mathbb{E}((X_1 - \mu)^2)/(n^2) \leq \\
    & \leq 2p + 1/\mu \rightarrow 0
\end{align*}
\end{proof}

