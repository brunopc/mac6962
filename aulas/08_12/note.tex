\section{Aula 12 de Agosto de 2019}
\label{2019_08_12}
\subsection{Elementos distintos em um stream}
Considere uma sequência $\sigma = (s_1, s_2, ..., s_m)$ com $s_i \in [n]$ e $m$ e $n$ muito grandes. Além disso, considere que devido ao tamanho essa sequência não pode ser armazenada em RAM. Nessa seção queremos responder quantos elementos diferentes ocorrem em $\sigma$, denotamos esse valor por $\Delta$. Seja $X = (x_i)_{i \in [n]}$ em que $x_i$ é o número de ocorrências de $i$ em $\sigma$, podemos interpretar a pergunta descrita acima como o número de elementos de X em que $x_i>0$.


Um exemplo de aplicação seria um servidor web com muitos acessos. O $\sigma$ seriam os IPs dos clientes que acessam o servidor e queremos saber quantos usuários o serviço tem.


Queremos resolver esse problema consumindo memória polilogarítmica $\mathcal{O}((\log{}n)^{\mathcal{O}(1)})$. Para isso, usaremos novamente a ideia de aleatorização e estimaremos um $\Delta$ aproximado. Denotaremos esse algoritmo por $\mathcal{B}$.


Primeiramente, vamos tentar resolver um problema mais simples. Queremos construir um algoritmo $\mathcal{A}$ tal que dado $T \in \mathbb{R}$ temos:
\begin{enumerate}
    \centering
    \item Se $\Delta \geq (1+\epsilon)T$, então $\mathbb{P}(\mathcal{A}(\sigma) = 1) \geq 1 - \delta$
    \item Se $\Delta \leq (1-\epsilon)T$, então $\mathbb{P}(\mathcal{A}(\sigma) = 0) \geq 1 - \delta$
\end{enumerate}
em que $\epsilon$ e $\delta$ são números reais e representam o erro na aproximação e a probabilidade de um resultado falso, $\mathcal{A}(\sigma)$ é o valor de retorno do algoritmo quando aplicado em $\sigma$ e $\mathbb{P}(B)$ é a probabilidade do eventro $B$ ocorrer.


Antes de descrever $\mathcal{A}$ vamos descrever um algoritmo $\mathcal{A}'$ que será usado como subrotina de $\mathcal{A}$.

\subsubsection{Algoritmo $\mathcal{A}$'}O algoritmo $\mathcal{A}'$ funciona da seguinte forma. Escolha um conjunto $\mathcal{S} \subseteq [n]$ com $\mathbb{P}( i \in \mathcal{S}) = \frac{1}{T}$ para todo $i \in [n]$, eventos independentes.

Podemos interpretar a ocorrência de um elemento $s_j = k$ de $\sigma$ como $x_k \leftarrow x_k + 1$, isto é, crescemos o um contador que marca a frequência de cada elemento, começamos com $X$ vazio. 

O algoritmo $\mathcal{A}$' mantém um "esboço" de $X$, a saber:
$$\sigma(S) = \sum_{i \in \mathcal{S}}x_i$$

$\mathcal{A}$' pode retornar os seguintes valores:
\[
   \mathcal{A}'(\sigma) = \begin{cases} 
                    1,\ \text{se }\sigma(\mathcal{S}) > 0\\ 
                    0,\ \text{c.c }\end{cases}
\]


Seja $\mathbb{P}_0 = \mathbb{P}(\sigma(\mathcal{S})= 0)$ a probabilidade de $\mathcal{S}$ não conter nenhum elemento de $\sigma$, temos que:
\begin{equation*}
\mathbb{P}_0 = \left(1-\frac{1}{T} \right)^\Delta    
\end{equation*}
supondo $\Delta \geq (1+\epsilon)T$, temos:
\begin{align*}
    \mathbb{P}_0 \leq \exp(-\frac{\Delta}{T}) &\leq \exp(-1-\epsilon) \leq \frac{1}{e} \left(1-\frac{\epsilon}{2}\right)\\
    \mathbb{P}_0 &\leq \frac{1}{e} \left(1-\frac{\epsilon}{2}\right)
\end{align*}
agora, supondo $\Delta \leq (1-\epsilon)T$, temos:
\begin{align*}
    \mathbb{P}_0 \geq \exp(-\frac{1}{T}-\frac{1}{T^2})^\Delta &\geq \exp(-1-\epsilon-T^{-1}) = \frac{1}{e}\frac{\exp(\epsilon)}{\exp(T^{-1})}\\
     \frac{1}{e}\frac{\exp(\epsilon)}{\exp(T^{-1})} \geq \frac{1}{e}\frac{\exp(\epsilon)}{\exp(\frac{\epsilon}{2})} &=\frac{1}{e}\exp(\frac{\epsilon}{2}) \geq \frac{1}{e}\left(1+\frac{\epsilon}{2}\right)
\end{align*}
para $T \geq \frac{2}{\epsilon}$.

\subsubsection{Algoritmo $\mathcal{A}$}Agora, estamos pontos para descrever o algoritmo $\mathcal{A}$. Seja $k =\frac{2e^2}{\epsilon^2}\log{\frac{1}{\delta}}$, o algoritmo $\mathcal{A}$ consiste em executar $\mathcal{A}$' $k$ vezes. 


Seja $z \in \mathbb{N}$ o número de vezes que $\mathcal{A}'(\sigma)=0$. Temos os seguintes valores para $\mathcal{A}$.

\[
   \mathcal{A}(\sigma) = \begin{cases} 
                    1,\ \text{se }z \leq \frac{k}{e}\\ 
                    0,\ \text{c.c}\end{cases}
\]

\subsubsection{Análise de $\mathcal{A}$}
Vamos calcular a probabilidade com que $\mathcal{A}$ devolve uma resposta errada.


Suponha $\Delta \geq \left(1+\epsilon\right)T$, então:
\begin{equation*}
E(z) = \mathbb{P}_0 k \leq \frac{1}{e}\left(1-\frac{\epsilon}{2}\right)k    
\end{equation*}
para esse $\Delta$, obtemos uma resposta errada se $z > \frac{k}{e}$ e temos que:
\begin{equation*}
\frac{k}{e} = \frac{k}{e}\left(1-\frac{\epsilon}{2}\right) + \frac{k\epsilon}{2e} \geq E(z) + \frac{k\epsilon}{2e}    
\end{equation*}
com isso, obtemos o seguinte resultado:
\begin{align*}
    \mathbb{P}\left(z>\frac{k}{e}\right) &\leq \mathbb{P}\left(z > E(Z)+\frac{k\epsilon}{2e}\right),\text{ aplicando McDiarmid}\\
    &\leq \exp\left(-\frac{\epsilon^2k}{2e^2}\right), \text{ substituindo $k$}\\
    &\leq \delta        
\end{align*}    
Portanto, a probabilidade de erro para $\Delta \geq \left(1+\epsilon\right)T$ é inferior a $\delta$, como era desejado.

Agora, suponha que $\Delta \leq (1-\epsilon)T$. Podemos realizar um processo equivalente ao feito acima, temos que:
\begin{equation*}
E(z) = \mathbb{P}_0 k \geq \frac{1}{e}\left(1+\frac{\epsilon}{2}\right)k    
\end{equation*}
assim, temos uma resposta errada se $z \leq \frac{k}{e}$, e:
\begin{equation*}
\frac{k}{e} = \frac{k}{e}\left(1+\frac{\epsilon}{2}\right) - \frac{k\epsilon}{2e} \geq E(z) - \frac{k\epsilon}{2e}    
\end{equation*}
com isso, obtemos o seguinte resultado:
\begin{align*}
    \mathbb{P}\left(z \leq \frac{k}{e}\right) &\leq \mathbb{P}\left( z \leq E(Z) - \frac{k\epsilon}{2e}\right),\text{ aplicando McDiarmid}\\
    &\leq \exp\left(-\frac{\epsilon^2k}{2e^2}\right), \text{ substituindo $k$}\\
    &\leq \delta        
\end{align*}
Portanto, a probabilidade de erro para $\Delta \leq (1-\epsilon)T$ é inferior a $\delta$. Com isso, vemos que $\mathcal{A}$ cumpre ambas as condições estabelecidas no inicio da seção.


$\mathcal{A}$ distingue entre $\Delta \geq (1+\epsilon)T$ e $\Delta \leq (1-\epsilon)T$. Suponha $T_0 =\frac{2}{\epsilon}$ e suponha $l$ tal que:
\begin{equation*}
    T_0(1 + 2\epsilon)^{l-1} \leq n \leq T_0(1 + 2\epsilon)^{l}
\end{equation*}
temos que $l \leq \frac{\log(n)}{\log(T_0(1 + 2\epsilon))} \leq \frac{\log(n)}{2}$. Além disso, suponha o seguinte caso:
\begin{align*}
    \Delta &\leq T_0(1 + 2\epsilon)^l\\
   &\leq \frac{1}{1+2\epsilon}T_0(1+2\epsilon)^{l+1}\\ 
   &\leq (1-\epsilon)T_0(1+2\epsilon)^{l+1}
\end{align*}
pois
\begin{align*}
    \frac{1}{1+2\epsilon} &\leq 1 - \epsilon, \quad (\epsilon \leq \frac{1}{2})\\
\end{align*}
logo, a chamada de $\mathcal{A}$ para esses valores devolve:
\begin{align*}
    &T_0(1+2\epsilon)^{l - 2}, \quad \mathcal{A}(\sigma = 1)\\
    &T_0(1+2\epsilon)^{l - 1}\\
    &T_0(1+2\epsilon)^{l},\quad \text{Valor mais próximo de $\Delta$}\\
    &T_0(1+2\epsilon)^{l + 1}, \quad \mathcal{A}(\sigma = 0)\\
\end{align*}
Então, vemos que o erro máximo dista de $(1 + 2\epsilon)^2 \leq 1 + 5\epsilon$ de $\Delta$. Portanto, a saída do algoritmo é o primeiro $T$ tal que $\Delta = T(1 \pm 5\epsilon)$ com $\mathcal{A}(\sigma)=0$.
\subsubsection{Algoritmo $\mathcal{B}$}Esse é o algoritmo final. Primeiro, executamos $\mathcal{A}$ com vários valores $T=T_0,...$. Vimos que $\Delta=(1 \pm 5\epsilon)T$ com probabilidade $\mathbb{P} \leq L\delta=\eta$.

Precisamos usar $\delta=\frac{\eta}{L}\geq\frac{\eta \epsilon}{\log{n}}$ para
\begin{equation*}
\log{\frac{1}{\delta}}=\log{\log{n}}+\log{\frac{2}{\epsilon}}+\log{\frac{1}{\eta}}
\end{equation*}

\subsubsection{Comentários sobre implementação}
Uma possibilidade na hora de implementar o algoritmo é não armazenar $S$. Podemos usar uma função de hashing $h:[n] \rightarrow [j]$ e definir:
\begin{align*}
    S = h^{-1}(1) = \{ i \in [n]: h(i) = 1\}
\end{align*}
Mais informações podem ser encontrada no capítulo 6 do Foundations of Data Science.


Por fim, nessa aula também foi enunciada a desigualdade de Chernoff.
\begin{lema}[Desigualdade de Chernoff]
 Sejam $X_1,...,X_n$ variáveis aleatórias independentes com $\mathbb{P}(X_i = 1) = p$ e seja:
$$Y = \sum_{i=1}^n X_i$$ 
Para todo $t\geq 0$, temos:
\begin{enumerate}
    \centering
    \item $\mathbb{P}(Y \leq E(Y)-t) \leq \exp(-\frac{2t^t}{n})$
    \item $\mathbb{P}(Y \geq E(Y)+t) \leq \exp(-\frac{2t^t}{n})$
\end{enumerate}
\end{lema}

A desigualdade McDiarmid generaliza a desigualdade de Chernoff.