\section{Aula 21 de Novembro de 2019}
\label{2019_11_21}

No método de Monte Carlo visto anteriormente, considerávamos 
amostragens de uma distribuição uniforme do espaço de medida. Agora
veremos que a condição de uniformidade pode ser relaxada um pouco.

\begin{definicao}
	Seja $\Omega$ conjunto finito, $\varepsilon > 0$ e $\cA$ que gera 
	elementos aleatórios $\omega$ de $\Omega$. Dizemos que $\cA$ fornece
	uma amostra \emph{$\varepsilon$-uniforme} se, para todo 
	$S\subset \Omega$, temos 
	$\left|\PP(\omega \in S) - \frac{|S|}{|\Omega|}\right|\leq \varepsilon$
\end{definicao}

Mais geralmente, temos $X \mapsto \Omega(X) \subseteq \Omega$ e 
queremos que $\cA(X) = \omega$ seja uma amostra~$\varepsilon$-uniforme
de $\Omega(X)$:
\begin{equation*}\
\forall S\subset \Omega(X), \;
\left| \PP(\omega \in \Omega) - \frac{|S|}{|\Omega|} \right|
\end{equation*}

Por exemplo, a aplicação $G\mapsto I(G) = \{\text{conjuntos independentes de} G\}$.
Estamos interessados em algoritmos que fazem essa amostragem rapidamente:
\begin{definicao}[FPAUS]
	Dizemos que um algoritmo $\cA$ é um \emph{Fully Polinomial Almost 
	Uniform Sampler} (FPAUS) se $\omega = \cA(X)$ é uma amostra
	$\varepsilon$-uniforme de $\Omega(X)$, e tem tempo polinomial
	de execução em $|X|$ e $|\log (1/\varepsilon)|$.
\end{definicao}
Algo desejável também são resultados da forma $\exists FPAUS \implies \exists FPRAS$.

\subsection{Contagem de conjuntos independentes em grafos}
Seja $G = G^n$ um grafo fixo. Sejam $e_1, \dots, e_m$ suas arestas, e 
$G_i = G[e_1, \dots, e_i]$. É fácil ver que $G_0$ = grafo com $n$ 
vértices isolados e $G_m = G$.

Seja $\Omega(G_i) = I(G_i) = \{ \text{conjuntos independentes de} G_i \}$.
Temos, pela série telescópica,
\begin{equation*}\label{key}
|G(\Omega)| = \frac{|\Omega(G_{m})|}{|\Omega(G_{m-1})|} \cdot \frac{|\Omega(G_{m-1})|}{|\Omega(G_{m-2})|} \cdot \cdots \cdot \frac{|\Omega(G_{1})|}{|\Omega(G_{0})|}\cdot|\Omega(G_0)|
\end{equation*}
Vamos encontrar boas aproximações para $r_i = \frac{|\Omega(G_i)|}{|\Omega(G_{i-1})|}.$ com $i = 1, \dots, m$.
Usaremos o FPAUS para gerar amostrar quase uniformes em $\Omega(G_{i-1})$. Note que $G_{i-1} \subset G_i$ mas 
$\Omega(G_{i}) \subset \Omega(G_{i-1})$. Obteremos uma aproximação
$\tilde{r}_i$ para $r_i$. A saída do algoritmo será então
$2^n\prod_{1 \leq i \leq m}\tilde{r}_i$.

Precisamos comparar esse valor com $|\Omega(G)| = 2^n\prod_{1 \leq i \leq m}r_i$. 
Vamos estimar $R:= \prod_{1 \leq i \leq m} \frac{\tilde{r}_i}{r_i}$.

Para termos uma $(\varepsilon, \delta)$-aproximação de $|\Omega(G)|$,
queremos
\begin{equation*}\label{key}
	\PP\left(\left|2^n\prod_{1 \leq i \leq m}\tilde{r}_i - |\Omega(G)| \right| \geq \varepsilon|\Omega(G)|\right) leq \delta
\end{equation*}
isto é, dividindo por $|\Omega(G)|$, basta que
\begin{equation}\label{Rprob}
	\PP(|R-1| \geq \varepsilon) \leq \delta
\end{equation}
\begin{lema}
	Suponha que $\tilde{r}_i$ é uma $(\varepsilon/2m, \delta/m)$-aproximação de $r_i$.
	Então \ref{Rprob} vale.
\end{lema}
\begin{proof}
	Temos que $\PP(|\tilde{r}_i - r_i|\geq \frac{\varepsilon}{2m}r_i)\leq \delta$
	para todo $i = 1, \dots, m$. Assim, 
	${\PP(\exists i : |\tilde{r}_i - r_i|\geq \frac{\varepsilon}{2m}r_i) \leq \delta}$.	 
	Temos que, com probabilidade pelo menos $1-\delta$
	$$ 1 - \frac{\varepsilon}{2m} \leq \frac{\tilde{r}_i}{r_i} \leq 1 + \frac{\varepsilon}{2m} \; \forall i = 1, \dots, m.$$
	Por fim, com probabilidade pelo menos $1 - \delta$ temos
	$$1-\varepsilon \leq \left(1 - \frac{\varepsilon}{2m}\right)^m \leq R = \prod_{1 \leq i \leq m}\frac{\tilde{r}_i}{r_i} \leq \left(1 + \frac{\varepsilon}{2m}\right)^m\leq e^{\varepsilon/2} \leq 1 + \varepsilon. \;(\varepsilon \leq \varepsilon_0).$$
	como queríamos demonstrar.
\end{proof}
\subsection{Algoritmo para estimativa de $r_i$}

Segue abaixo o algoritmo para a estimativa de $r_i$:

\textbf{Entrada:} Grafos $G_i$ e $G_{i-1}$.

\textbf{Saída:} $\tilde{r}_i$.
\begin{enumerate}
	\item $X \leftarrow 0$;
	\item Repita $M = \lceil 1296 m^2\varepsilon^{-2}\log (2m/\delta)\rceil$;
	\subitem(a) gere uma amostra $I$ $(\varepsilon/(6m))$-uniforme de $\Omega(G_{i-1})$;
	\subitem(b) Se a amostra $I$ é independente em $G_i$, faça $X \leftarrow X+1$;
	\item Devolva $\tilde{r}_i = X/M$.
\end{enumerate}
\begin{lema}
	A saída do algoritmo acima é uma $(\varepsilon/2m, \delta/m)$-aproximação de $r_i$.
\end{lema}
\begin{proof}
	Começamos provando que 
	$r_i = \frac{|\Omega(G_{i-1})|}{|\Omega(G_i)|} \geq 1/2$. Basta
	considerar a injeção $\Omega(G_{i-1}) - \Omega(G_i) \to\Omega(G_i)$,
	$I\mapsto I - \text{ o ``menor'' extremo de } e_i$.
	Sejam $I_1, \dots, I_n$ as amostras geradas em (a) pelo algoritmo.
	Seja $X_i = \mathds{1}\{I_j \in \Omega(G_i)\}$. Pela hipótese da 
	$(\eps/(6m))$-uniformidade, temos
	$$\left|\PP(X_i = 1) - \frac{|\Omega(G_i)|}{|\Omega(G_{i-1})|} \right| \leq \frac{\eps}{6m}$$
	para todo $i$. Mas $\EE(X_i) = \PP(X_i).$
	Somando para $1 \leq i \leq M$, temos
	\begin{align*}
	\left| \frac{1}{M} \EE(X) - r_i\right|\leq & \frac{\eps}{6m}, 
	\text{ mas como }\EE(X) = \EE(\tilde{r}_i) \\
	\left| \frac{1}{M} \EE(\tilde{r}_i) - r_i\right|\leq & \frac{\eps}{6m}.
	\end{align*}
	
	Note que $\EE(\tilde{r}_i) \geq r_i - \frac{\eps}{6m} \geq \frac{1}{2} - \frac{\eps}{6} \geq \frac{1}{3}$.
	
	Pelo teorema da Aproximação de Monte Carlo, temos que se 
	$M \geq \frac{3\cdot 3}{(\eps/(12m))^2}\cdot \log (2m/\delta)$, 
	então $\tilde{r}_i$ é uma $(\eps/(12m), \delta/m)$-aproximação
	de $r_i$, i.e., $\PP(|\tilde{r}_i/\EE(\tilde{r}_i) - 1| \geq \eps/(12m)) \leq \delta/m.$ Equivalentemente, com probabilidade pelo menos $1 - \delta/m$ temos
	\begin{equation}\label{eq1}
		1 - \frac{\eps}{12m} \leq \frac{\tilde{r}_i}{\EE(\tilde{r}_i)}
		\leq 1 + \frac{\eps}{12m}
	\end{equation}
	Como $|\EE(\tilde{r}_i) - r_i| \leq \eps/(6m)$, temos
	\begin{equation}\label{eq2}
		1 - \frac{\eps}{3m} \leq 1 - \frac{\eps}{6mr_i} \leq
		\frac{\EE(\tilde{r}_i)}{r_i} \leq 1 + \frac{\eps}{6mr_i} \leq
		1 + \frac{\eps}{3m},
	\end{equation}
	e usando que $r_i \geq 1/2$, e temos, multiplicando \ref{eq1} e \ref{eq2}:
	\begin{equation*}
		1 - \frac{\eps}{2m} \leq \left(1 - \frac{\eps}{3m} \right)
		\left(1 - \frac{\eps}{12m}\right) \leq \frac{\tilde{r}_i}{r_i}
		\leq \left(1 + \frac{\eps}{3m}\right)
		\left(1 + \frac{\eps}{12m}\right) \leq 1 + \frac{\eps}{2m}.
	\end{equation*}
\end{proof}

\subsection{Conclusão}
Concluímos que uma amostragem $\eps/6m$-uniforme implica em uma
$(\eps, \delta)$-aproximação para o problema $\#I(G)$

\begin{pergunta}
Existe um FPAUS para o problema $\#I(G)$?
\end{pergunta}

\textbf{Sim}, para grafos com grau máximo $\Delta \leq 4$.

\textbf{Não}, para grafos com grau máximo $\Delta = 2000$ a menos que $RP = NP$.