\documentclass[a4paper,12pt]{article}

\usepackage[brazilian]{babel}
\usepackage[utf8]{inputenc}
\usepackage[T1]{fontenc}
\usepackage{amsmath}
\usepackage{amssymb}

\begin{document}

\section{Aula 21 de Outubro de 2019}
\label{2019_10_21}

Da aula passada temos que:

\[\frac{1}{H}\geq\epsilon>0 ,\ p=\frac{1+\epsilon}{n}\]

Então $\mathbb{P}(L_1(G(n,p))\geq (1/24)\epsilon^2 n)$ converge para 1 quando $n\rightarrow\infty$.\\

Executamos $t=sn^2$ passos da busca em profundidade, onde $\sigma=\epsilon/4$, e paramos.Examinamos a estrutura neste momento.\\

Vamos provar que quase certamente $|F|\geq n\epsilon^2/24$ sendo que $L_1(G_p) \geq n\epsilon^2/24$, quase certamente.\\
Podemos supor que o número de arestas descobertas até este momento é tal que, com alta probabilidade,
\[(*)\sigma(1+\frac{2\epsilon}{n})n\leq A\leq \sigma (1+2\epsilon)\]

De fato $\mathbb{E}(A)=tp=\sigma n (1+\epsilon)$ e $Ap_i(t,p)$ e * ocorre quase certamente. Suponha que * ocorre e $\epsilon \leq\ 1/4 $.

\begin{fato}
$ |E|<2sn $
\end{fato}
\begin{proof}
Suponha o contrário. Considere o momento em que $ |E|=2sn$. Temos o seguinte:
\[|F|\leq1+A\leq1+\sigma(1+2\epsilon)n<2\sigma n\]
Dessa forma temos que:
\[\sigma n^2 = t \geq |E||U|\geq2\sigma n(n-|E|-|F|) \geq 2\sigma n (n-2\sigma n - 2\sigma n ) = 2 \sigma n^2 (1-4\sigma)\]

O que é uma contradição.\\
\end{proof}

\begin{fato}
 $|F|\geq(1/6)\epsilon\sigma n = (1/24)\epsilon ^2 n$
\end{fato}
\begin{proof}
Suponha o contrário, ou seja, que $|F|<(1/6)\epsilon\sigma n$.
Temos que $|E|+|F|\geq A = \delta (1+\frac{\epsilon}{3})n$, portanto, usando o resultado acima:

\[|E|>\sigma(1+2/3\epsilon)n-(1/6)\epsilon\sigma n = \sigma(1+\epsilon/2)n\]

Temos assim que $\sigma(1+\epsilon/2)n<|E|<2\sigma n$.\\

Sabemos, porém, que:

\[|E||U|-|E|(n-|E|-|F|)\]
\[\geq\sigma(1+\epsilon/2)n(n-\sigma(+\epsilon/2)n-(1/6)\epsilon\sigma n)\]
\[=\sigma n^2 (1+\epsilon/2)(1-\sigma-2/3\epsilon\sigma)\]
\[\geq\sigma n^2(1+\epsilon/6-\epsilon^2/6)\]
\[>\sigma n^2\]

Como $|E||U|\leq\sigma n^2$, chegamos em uma contradição.
\end{proof}

\begin{teorema}
Suponha que $0<\epsilon\leq1/2$, $p=(1+\epsilon/n)$. Então $\lim_{n\to\infty} \mathbb{P}(L_2(G_p)>(200/\epsilon^3)\log n)=0$
\end{teorema}
\begin{proof}
Para a prova, usaremos exploração em duas rodadas. Seja $p_1 = (1+\epsilon/2)/n$ e $p_2$ tal que $(1-p)=(1-p_1)(1-p_2)$.\\

Então $G(n,p)=G(n,p_1)\cup G(n,p_2)$, independentes. Temos $p=p_1+p_2-p_1p_2\leq p_1+p_2$, o que implica que  $p_2\geq\epsilon/2n$.\\

Sabemos que quase certamente $L_1(G,n,p_1))\geq(1/24)(\epsilon^2/4)n = (1/96)\epsilon^2n$.\\

Seja $C$ uma componente de $G(n,p_1)$ com $|V(C)|=L_1(G(n,p_1))$. Adicione as arestas de $G(n,p_2)$ com ambas as pontas em $V(G(n,p_1))\backslash V(C)$ ou $G(n,p_1)$, criando um novo grafo.\\ 

Seja $C'$ uma componente contida em $V(G(n,p_1))\backslash V(C)$ neste novo grafo.
Suponha que $|C'|\geq (200/\epsilon^3)\log n$. Adicione agora as arestas de $G(n,p_2)$ entre $V(C')$ e $V(C)$. A probabilidade de nenhuma aresta aparecer entre $C'$ e $C$ é:\\
\[(1-p_2)^{|V(C)||V(C')|}\]
\[\leq e^{-p_2|V(C)||V(C')|}\]
\[\leq  e^{-\frac{\epsilon}{2n}}\frac{1}{96}\epsilon^2 n \frac{200}{\epsilon^3}\log n\]
\[=n^{-\frac{-2\epsilon}{27}}\rightarrow\]

Como o número de tais $C'$ é menor ou igual a n, temos pela cota da união que a probabilidade de que alguma delas não é "absorvida" por $C$ é menor ou igual a $n^{-\frac{1}{24}}$, o que tende a zero, como queríamos demonstrar.\\

\end{proof}

\end{document}
