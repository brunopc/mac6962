\documentclass[a4paper,12pt]{article}

\usepackage[brazilian]{babel}
\usepackage[utf8]{inputenc}
\usepackage[T1]{fontenc}
\usepackage{amsmath}
\usepackage{amssymb}

\begin{document}

\section{Aula 21 de Outubro de 2019}
\label{2019_10_21}

Relembrando:

\[\frac{1}{H}\geq\epsilon>0 ,\ p=\frac{1+\epsilon}{n}\]

Então:

\[\mathbb{P}(L_1(G(n,p))\geq\frac{1}{24}\epsilon^2 n)\]

Converge para 1 quando $n\rightarrow\infty$.\\
\\
P.S: (relembrando)\\
\\
Executamos $t=sn^2$ passos da busca em profundidade, onde $\sigma=\epsilon/4$, e paramos.\\
Examinamos a estrutura neste momento. Vamos provar que quase certamente $|F|\geq n\epsilon^2/24$ sendo que $L_1(G_p) \geq n\epsilon^2/24$, quase certamente.\\
Podemos supor que o $n^o$ de arestas descobertas até este momento é tal que, com alta probabilidade,
\[(*)\sigma(1+\frac{2\epsilon}{n})n\leq A\leq \sigma (1+2\epsilon)\]

De fato $\mathbb{E}(A)=tp=\sigma n (1+\epsilon)$ e $Ap_i(t,p)$ e * ocorre quase certamente. Suponha que * ocorre e $\epsilon \leq\ 1/4 $.

\subsection{Fato $ |E|<2sn $}

$\rightarrow$ suponha o contrário, considere o momento em que $ |E|=2sn$. Temos o seguinte:
\[|F|\leq1+A\leq1+\sigma(1+2\epsilon)n<2\sigma n\]
Assim 
\[\sigma n^2 = t \geq |E||U|\geq2\sigma n(n-|E|-|F|) \geq 2\sigma n (n-2\sigma n - 2\sigma n ) = 2 \sigma n^2 (1-4\sigma)\]

Contradição.\\
\\

Provaremos agora que $|F|\geq(1/6)\epsilon\sigma n = (1/24)\epsilon ^2 n$. Assim $|E|>\sigma(1+2/3\epsilon)n-(1/6)\epsilon\sigma n = \sigma(1+\epsilon/2)n$.\\
Temos assim que $\sigma(1+\epsilon/2)n<|E|<2\sigma n$, mas $|E||U|-|E|(n-|E|-|F|)\geq\sigma(1+\epsilon/2)n(n-\sigma(+\epsilon/2)n-(1/6)\epsilon\sigma n)=\sigma n^2 (1+\epsilon/2)(1-\sigma-2/3\epsilon\sigma)\geq\sigma n^2(1+\epsilon/6-\epsilon^2/6)>\sigma n^2$, mas $|E||U|\leq\sigma n^2$, contradição.



\subsection{Teorema: suponha $0<\epsilon\leq1/2$, $p=(1+\epsilon/n)$. Então $\lim_{n\to\infty} \mathbb{P}(L_2(G_p)>(200/\epsilon^3)\log n)=0$}

Prova: Usamos exploração em duas rodadas. Seja $p_1 = (1+\epsilon/2)/n$, $p_2$ tal que $(1-p)=(1-p_1)(1-p_2)$.\\
Então $G(n,p)=G(n,p_1)\cup G(n,p_2)$, independentes. Temos $p=p_1+p_2-p$, $p_2\leq p_1+p_2\rightarrow p_2\geq\epsilon/2n$.\\

Sabemos que quase certamente $L_1(G,n,p_1))\geq(1/24)(\epsilon^2/4)n = (1/96)\epsilon^2n$.\\

Seja $C$ uma componente de $G(n,p_1)$ com $|V(C)|=L_1(G(n,p_1))$. Adicione as arestas de $G(n,p_2)$ com ambas as pontas em $V(G(n,p_1))\backslash V(C)$ ou $G(n,p_1)$. Seja $C'$ uma componente contida em $V(G(n,p_1))\backslash V(C)$ neste novo grafo.\\
Suponha que $|C'|\geq (200/\epsilon^3)\log n$. Adicione agora as arestas de $G(n,p_2)$ entre $V(C')$ e $V(C)$. A probabilidade de nenhuma aresta aparecer entre $C'$ e $C$ é:\\

\[(1-p_2)^{|V(C)||V(C')|}\leq e^{-p_2|V(C)||V(C')|}\leq  e^{-\frac{\epsilon}{2n}}\frac{1}{96}\epsilon^2 n \frac{200}{\epsilon^3}\log n=n^{-\frac{-2\epsilon}{27}}\rightarrow 0  \]

\begin{flushright}
$\square$
\end{flushright}\\

Vamos agora para o caso $p=(1-\epsilon)/n$. Temos que $(1-\epsilon)/n<(1-\epsilon)/(n-1)$.
Estudaremos o comportamento da busca em largura no $G(n,p)$.\\
Fixe o vértice inicial da busca $v=v_1$. A partir de $v_1$, descobrimos os vértices $v_2,v_3,..,v_k,...$ que pertencem à componente $C_v$ de v. Seja $z_0=1$ e $z_i =$ # de novos vértices de $C_v$ descobertos pelo vértice $v_i$ na busca em largura.\\
\\
Afirmação: Se $\sum_{i=1}^{k-1}z_i<k-1$, então $|V(C_u)|<k$.\\
\\
Prova: Suponha $|V(C_u)|\geq k$: $v=v_1,v_2,...,v_k,...$, $k-1\leq z_1+...+z_k$.\\

\begin{flushright}
$\square$
\end{flushright}\\

Portanto,
 \[\mathbb{P}(|V(C_v)|\geq k)\leq \mathbb{P}(\sum_{i=1}^{k-1}z_i\geq k-1).  \]
 
 Sejam $Y_1,Y_2,...,Y_{k-1}$ variáveis aleatórias independentes, todas com distribuição $ B_i(n-1,(1-\epsilon)/(n-1))$. Um argumento de acoplamento mostra que:
 
 \[\mathbb{P}(\sum_{i=1}^{n-1}z_i\geq T)\leq\mathbb{P}(\sum_{i=1}^{n-1}y_i\geq T)=\mathbb{P}(B_i((k-1)(n-1),(1-\epsilon)/(n-1))\geq T) \]
 
 
 Para $T=k-1$, $\mathbb{P}(\sum z_i \geq T)\leq e^{-(1/2)\epsilon^2(k-1)}$.


\end{document}
