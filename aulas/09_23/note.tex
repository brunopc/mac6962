\section{Aula 23 de Setembro de 2019}
\label{2019_09_23}

\subsection{Gaussianas em Espaços de Dimensão Alta}

Suponha que temos variáveis aleatórias ${ X_1, X_2, \dots, X_d \sim N(0,1)}$
independentes, e considere $N_d(0,1) \sim X = (X_1, X_2, \dots, X_d) \in \RR^d$.
Como cada $X_i$ possui função densidade de probabilidade (f.d.p.) 
$\phi(x) = \frac{1}{\sqrt{2\pi}}e^{-x^2/2}$, então a f.d.p. de $X$ será
\begin{equation*}\label{key}
\phi(X) := \frac{1}{(2\pi)^{d/2}}e^{\frac{-\sum_{i = 1}^{d}{X_i^2}}{2}}
= \frac{1}{(2\pi)^{d/2}}e^{\frac{-\|X\|^2}{2}}
\end{equation*}

De forma mais geral, podemos considerar distribuições em $\RR^d$ que são
imagens afins da gaussiana esférica. Por simplicidade vamos apenas considerar
gaussianas com centro~${\mu = (\mu_1, \dots, \mu_d)}$ 
e~$\sigma = (\sigma_1, \dots, \sigma_d)$ com $\sigma_1 = \dots = \sigma_d$.

\begin{pergunta}
	Onde está concentrada maior parte da massa de $\phi(X)$?
\end{pergunta}
Por exemplo, para $d = 1$, a massa de $\phi(X)$ está concentrada no intervalo
de comprimento $O(1)$ en torno da média. E se $d$ é grande?

Apesar da média de $X$ ser na origem, devemos lembrar que o volume da esfera
está concentrado em um intervalo constante ao redor de uma casca esférica:
\begin{equation*}\label{key}
V := V(d) = \Vol(B^d) = \frac{\pi^{d/2}}{\frac{d}{2}\Gamma(\frac{d}{2})} 
   = (1 + o(1))(\frac{2\pi e}{2})^{d/2}\frac{1}{\sqrt{d\pi}}
\end{equation*}
com  $o(1)\ra 0$  quando $d\ra \infty$, assim

\begin{equation*}\label{key}
\Vol (rB^d) = r^d V 
= (1 + o(1))\left(\frac{2\pi e r^2}{d}\right)^{d/2}\frac{1}{\sqrt{d\pi}}.
\end{equation*}

Note que esse volume torna-se substancial quando $r = c\sqrt{d}$, para algum 
$c>0$ constante. Se~$X\sim N_d(0,1)$, então $X$ não está concentrado na
origem; veremos que $X$ fica concentrado no ``anel'' 
$\bigcup_{|t|\leq c}S^{d-1}(\sqrt{d-1} + t)$.

\subsection{Intuição}
Suponha $X\sim N_d(0,1)$. Então $\|X\|^2 = \sum_{i = 1}^{d}X_i^2$, e assim
$\|X\|^2$ é uma soma de variáveis aleatórias de mesma distribuição. Temos
\begin{equation*}\label{key}
\EE(\|X\|^2) = \sum_{i = 1}^d\EE(X_i^2) = d.
\end{equation*}
Ademais,
\begin{equation*}\label{key}
\Var(\|X\|^2) = \sum_{i = 1}^d\Var(X_i^2) 
= \sum_{i = 1}^d(\EE(X_i^4) - \EE(X_i^2)^2) = O(d).
\end{equation*}
É natural esperar que valha algo como
\begin{equation*}\label{key}
\PP(|\|X\|^2 - d| > t) \leq e^{-c_1 t^2/d}.
\end{equation*}

Da expressão acima, que pode ser demonstrada, segue que~$\|X\|^2$ 
está concentrada no intervalo~${[d - c_2\sqrt{d}, d + c_2\sqrt{d}]}$.

\subsection{Demonstração da concentração de $X$}

Vamos demonstrar a concentração de ${X\sim N_d(0,1)}$ no anel de raio
$\sqrt{d-1}$ e espessura $O(1)$ com contas explícitas. Queremos estudar

\begin{align*}
m_{a,b} & = \int_a^b\phi(r)\d(r^dV)& \\
		& = \int_a^b\frac{1}{(2\pi)^{d/2}}e^{-r^2/2}dr^{d-1}V\d r&\\
		& = \frac{dV}{(2\pi)^{d/2}}\int_a^b r^{d-1}e^{-r^2/2}\d r.&
\end{align*}
Sejam $g(r) := r^{d-1}e^{-r^2/2}$ e $f(r) := \log g(r) 
= (d-1) \log r - \frac{r^2}{2}$. Temos
\begin{align*}
f'(r)& = \frac{d-1}{r} - r,&\\
f''(r)& = -\frac{d-1}{r^2} - 1 < -1,&
\end{align*}
o que implica em $f'(\sqrt{d-1}) = 0$.
Assim, se $ r = \sqrt{d-1} + t$, temos que (por Taylor)
\begin{equation*}
f(r) = f(\sqrt{d-1}) + f'(\sqrt{d-1})t + \frac{1}{2}f''(\sqrt{d-1})t^2 +\dots
\end{equation*}
e, de fato, por uma outra versão de Taylor,
\begin{equation*}
f(r) = f(\sqrt{d-1}) + \frac{1}{2}f''(\rho)t^2,
\end{equation*}
onde $\rho = \rho(r)$ está entre $\sqrt{d-1}$ e $r = \sqrt{d-1} + t$.
Segue que 
\begin{equation*}
f(r) \leq f(\sqrt{d-1}) - \frac{1}{2}t^2,
\end{equation*}
e assim,
\begin{equation*}
g(r) \leq e^{f(\sqrt{d-1}) - \frac{1}{2}t^2} 
= g(\sqrt{d-1})e^{-\frac{1}{2}t^2}.
\end{equation*}

Fixe agora $c>0$, e seja $I = [\sqrt{d-1} - c, \sqrt{d-1} + c]$. 
Vamos mostrar que~${\int_{r \notin I}g(r)\d r << \int_0^{\infty}g(r)}$. Temos

\begin{align*}
\int_{r \notin I}g(r) \d r &\leq 
\int_{r \notin I}g(\sqrt{d-1})e^{-t^2/2} \d r&\\
& = g(\sqrt{d-1})\left( \int_{0}^{\sqrt{d-1} - c}e^{-t^2/2} \d r 
+ \int_{\sqrt{d- 1}+c}^{\infty}e^{-t^2/2} \d r \right),&\\
\end{align*}
e modificando os intervalos de integração ficamos com
\begin{align*}
\int_{r \notin I}g(r) \d r &
\leq g(\sqrt{d-1})\left( \int_{-\sqrt{d-1}}^{- c}e^{-t^2/2} \d r 
+ \int_{c}^{\infty}e^{-t^2/2} \d r \right),&\\
&\leq 2g(\sqrt{d-1})\int_c^{\infty}e^{-t^2/2}\d r&\\
\end{align*}
e terminamos usando o truque de multiplicar por $t/c > 1$ no 
intervalo de integração.
\begin{equation*}
2g(\sqrt{d-1})\int_c^{\infty}e^{-t^2/2}\d r \leq 
\frac{2}{c}g(\sqrt{d-1})\int_0^{\infty}te^{-t^2/2}
= \frac{2}{c}g(\sqrt{d-1})e^{-c^2/2}.
\end{equation*}

Vamos agora estimar $\int_0^{\infty}g(r)\d r$ por baixo. Note que 
se ${\rho \in [\sqrt{d-1}, \sqrt{d-1} + c/2}$, então
$${f''(\rho) = -\frac{d-1}{\rho^2}-1} \geq f''(\sqrt{d-1}) = -2.$$
Assim, temos $f(r)\geq f(\sqrt{d-1}) - t^2$ no intervalo 
${[\sqrt{d-1}, \sqrt{d-1} + c/2]}$, então
\begin{equation*}
g(r) = e^{f(r)} \geq g(\sqrt{d-1})e^{-t^2} \geq g(\sqrt{d-1})e^{-c^2/4}
\end{equation*}
neste intervalo, e
\begin{equation*}
\int_0^{\infty}g(r)\d r \geq \int_{\sqrt{d-1}}^{\sqrt{d-1} + c/2}
g(r)\d r \geq \frac{c}{2}g(\sqrt{d-1})e^{-c^2/4}.
\end{equation*}

Concluímos que 
\begin{equation*}
\frac{\int_{r \notin I}g(r) \d r}{\int_{0}^{\infty}g(r) \d r}
\leq \frac{4}{c^2}e^{-c^2/4}.
\end{equation*}

Provamos então o seguinte resultado:

\begin{teorema}
Sejam $X\sim N_d(0,1)$ e $c>0$. Então
$$\PP\left(\sqrt{d-1} - c \leq \|X\| \leq \sqrt{d-1} + c\right)\geq 1- \frac{4}{c^2}e^{-c^2/4}\qed$$
\end{teorema}

\subsection{Separação de Gaussianas em espaços de dimensão alta}

Seja $\phi_{\mu}(X) = \frac{1}{(2\pi)^{d/2}}e^{-(X-\mu)^2/2}$ a
gaussiana centrada em $\mu\in \RR^d$ e variância $1$. Suponha
que $X$ tem f.d.p. 
$$F(X) = \gamma\phi_{\mu_1}(X) + (1- \gamma)\phi_{\mu_2}(X),$$
com $\mu_1, \mu_2 \in \RR^d$ e $\gamma\in (0,1)$.

\begin{fato}[Dasgupta]
É possível identificar os parâmetros de $F$ se $\mu_1$ e $\mu_2$
estão suficientemente separados.
\end{fato}
Note que acabamos de ver que se a separação é de pelo menos $c\sqrt{d}$, 
o problema é fácil, veremos na próxima aula que $cd^4$ é
suficiente. Utilizando SVD, melhoramos essa cota para $c$.

